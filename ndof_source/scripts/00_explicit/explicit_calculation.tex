\documentclass{article}
\usepackage[utf8]{inputenc}

\title{N-DOF Calculation Outline}
\author{Shane W. Flynn}
\date{Feburary 19 2017}

\usepackage{graphicx}
\usepackage{amsmath}
\usepackage{braket}

\begin{document}

\maketitle

\section*{First Calculation}
For this project I am going to start at the simplest possible step. 
Mainly, I am going to calculate 
\begin{equation}
    \braket{0|G(x)|0}
\end{equation}
Here G(x) refers to a Gaussian distribution which we will assume (for now) is centered around 0 and has a standard deviation of 1. 
\begin{equation}
    G(x) = \frac{1}{\sqrt{2\pi \sigma^2}}e^{-\frac{(x-a)^2}{2\sigma^2}} \approx \sqrt{\frac{1}{2\pi}}e^{-\frac{x^2}{2}}
\end{equation}
The basis functions we are going to use are for the Harmonic Oscilator, we will consider the ground state diagonal element first.
Using McQuarrie P.225 for reference we find the wavefunctions (in terms of x, not using p, q space). 

\begin{equation}
    \ket{0} = \left (\frac{\alpha}{\pi}\right )^{\frac{1}{4}}e^{-\frac{\alpha x^2}{2}}
\end{equation}

Here we note that $\alpha = \frac{\sqrt{k\mu}}{\hbar}$ is always positive with no imaginary component, k is the spring constant and $\mu$ the reduced mass (for the calculation I will set things to 1 most likely, will need to be careful here). 

For the purpose of this first calculation we can simplify to write the integral as follows:
\begin{equation}
\begin{split}
      \braket{0|G(x)|0} =  \left (\frac{\alpha}{\pi}\right )^{\frac{1}{4}} \left (\frac{1}{2\pi}\right )^{\frac{1}{2}}      \left (\frac{\alpha}{\pi}\right )^{\frac{1}{4}} \braket{e^{-\frac{\alpha x^2}{2}}|e^{-\frac{x^2}{2}}|e^{-\frac{\alpha x^2}{2}}} \\
      = \frac{1}{\pi} \left (\frac{\alpha}{2}\right )^{\frac{1}{2}}\int  e^{-x^2(\alpha+\frac{1}{2})}dx
\end{split}
\end{equation}

\subsection*{Initial Outline}
\begin{enumerate}
    \item Set parameters and variables
    \item Construct the integral
    \item evaluate the integral using Scrambled Sobol Sequences in Fortran. 
    \item To do this we need to understand how to shuffle the sequence, and how many sobol points are necessary for the calculation.
\end{enumerate}
 
 \subsection*{Long-Term Code Outline}
\begin{enumerate}
    \item Calculate the $\braket{0|G(x)|0}$ case using a Hermite polynomial variable explicitly.
    \item Calculate the $\braket{0|G(x)|0}$ case using a Harmonic Oscillator variable explicitly.
    \item Write a subroutine to calculate the Hermite polynomials using the recursion relation for any input. 
    \item Write a subroutine to calculate the Harmonic Oscillator basis functions.
    \item Once we have a working system we can include a potential energy surface. 
\end{enumerate}

\section*{Generalized Information}
The Harmonic Oscillator wave-functions are given by 
\begin{equation}
    \psi_v(x) = \ket{v} = N_vHe_v(\sqrt{\alpha}x)e^{\frac{-\alpha x^2}{2}}
\end{equation}
Where the normalization constant $N_v = \frac{1}{\sqrt{2^vv!}}\left (\frac{\alpha}{\pi}\right )^{\frac{1}{4}}$. 
The Hermite Polynomials are part of the solution to the Harmonic Oscillator wave-function and are found by the following general formula:

\begin{equation}
    He = He_v(x) = (-1)^v e^{x^2}\frac{d^v}{dx^v}\left [ e^{-x^2} \right ]
\end{equation}
From evaluating we find the first few polynomials to be \begin{equation}
    \begin{split}
        H_0 = 1 \\
        H_1 = 2x \\
        H_2 = 4x^2 - 2 \\
        H_3 = 8x^3 - 12x
    \end{split}
\end{equation}

We also know that the Hermite polynomials obey the recursive relationship.
\begin{equation}
    H_{v+1} = 2xH_v(x) - 2vH_{v-1}(x)
\end{equation}

From this we see that we can write a subroutine to generate any of the Hermite Polynomials recursively. 
Once we can generate the Hermite Polynomials we can write another subroutine to call upon them, and to calculate the Eigenfunctions of the Harmonic Oscillator. 

In general we are going to construct the following matrix elements
\begin{equation}
    \braket{v|\sqrt{\frac{1}{2\pi}}e^{-\frac{x^2}{2}}|v}
\end{equation}
Considering our previous analysis on combinatorial scaling we should be able to generate a matrix with the above matrix elements for a 9 $\times$ 9 matrix.

\end{document}